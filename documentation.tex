%%%%%%%%%%%%%%%%%%%%%%%%%%%%%%%%%%%%%%%%%%%%%%%%%%%%%%%%%%%%%%%
% Copyright (C) 2016 Jahrme Risner.                           %
% Permission is granted to copy, distribute and/or modify     %
% this document under the terms of the GNU Free Documentation %
% License, Version 1.3 or any later version published by the  %
% Free Software Foundation; with no Invariant Sections, no    %
% Front-Cover Texts, and no Back-Cover Texts. A copy of the   %
% license is included in the section entitled                 %
% "GNU Free Documentation License".                           %
%%%%%%%%%%%%%%%%%%%%%%%%%%%%%%%%%%%%%%%%%%%%%%%%%%%%%%%%%%%%%%%

\documentclass[english,course,twocolumn]{notes}

\title{\texttt{notes} Documentation}
\author{Jahrme Risner}
\date{07}{22}{2016}

\begin{document}\clearpage

\section{Introduction}
The \verb|notes| document class is designed for taking notes.
These notes can be taken as either as a student organizing their course notes,
or by instructors to organize their lecture notes.

\begin{quote}
``Two words on what you get with this class file:
the main difference is the font (Palatino instead of Computer Modern);
the second main difference is the use of small-caps instead of boldface.
I have to say I needed these modifications because I was very tired of the standard, omnipresent \LaTeX\ style.
Other small notable differences are the centered titles--because I'm not writing a book--
and the theorem numbers placed before the word ``theorem''--to make easier searching for them.''
\\
---Stefano Maggiolo
\end{quote}

\section{Class Options}

The following options are available to customize the document class:
\begin{itemize}
\item Language:
    \begin{itemize}
    \item \verb|english| (default)
    \item \verb|italian|
    \end{itemize}
\item Structure:
    \begin{itemize}
    \item \verb|course|, a medium-length document structures with sections, subsections, and paragraphs (default)
    \item \verb|seminar|, a short document without structure
    \item \verb|talk|, the paper used while delivering a seminar
    \end{itemize}
\item Header Information:
    \begin{itemize}
    \item \verb|headertitle|
    \item \verb|headersection|
    \item \verb|headersubsection|
    \item \verb|headerno|
    \end{itemize}
\item Theorem Numbering:
    \begin{itemize}
    \item \verb|theoremsection|
    \item \verb|theoremsubsection|
    \item \verb|theoremnosection|
    \end{itemize}
\item Page Layout:
    \begin{itemize}
    \item \verb|onecolumn|, one column per page
    \item \verb|twocolumn|, two columns per page
    \item \verb|oneside|, margins and headers optimized for one-sided printing
    \item \verb|twoside|, margins and headers optimized for two-sided printing
    \item \verb|cleardoublepage|, sections begin on next \emph{verso} (right-hand) page
    \item \verb|nocleardoublepage|, sections continue where previous ended
    \end{itemize}
\end{itemize}

\section{Document Preamble}

The following fields are available to set document information:
\begin{itemize}
\item \verb|\title{TITLE}| (required)
\item \verb|\subject{SUBJECT}|
\item \verb|\speaker{SPEAKER}|
\item \verb|\place{PLACE}|
\item \verb|\author{AUTHOR}| (required)
\item \verb|\email{EMAIL@PLACE.COM}|
\item \verb|\date{DD}{MM}{YYYY}| (required)
\item \verb|\dateend{DD}{MM}{YYYY}|
\end{itemize}
Please note, the \verb|author| and \verb|email| refer to the name and email of the note taker.

\section{Recommended and Required Packages}

The following packackages are either required or highly recommended when using this document class:
\begin{itemize}
\item \verb|hyperref|, automates the creation of links in various locations and enables tweaking the PDF metadata.
\item \verb|mathtools|, enhancements to the standard AMS-\LaTeX\ packages; the documentation is an inspiring source of tips for beautiful typesetting.
\item \verb|booktabs|, standard \LaTeX\ tables look poorer than they should; learn why by reading booktabs' documentation.
\item \verb|multirow|, allow tables with cells spanning multiple rows.
\item \verb|fancyhdr|, allows customization of the headers and footers.
\item \verb|mparhack|, workaround for a bug in \LaTeX's \verb|\marginpar|.
\item \verb|tikz|, a high-level drawing language.
\item \verb|mathdots| redefines \verb|\dots| and its friends so that they change size when appropriate.
\item \verb|xfrac|, typesets nice inline fractions, as in \sfrac{1}{2}.
\item \verb|faktor|, same as \verb|xfrac|, but without shrinking the denominator's and numerator's size; useful for quotients.
\item \verb|cancel|, to draw diagonal bars through terms in an equation.
\end{itemize}

\section{New Commands}

Depending on the document style, we define several new commands.
For a \verb|course|, \verb|\lecture[duration]{dd}{mm}{yyyy}| writes the lecture information in the margin.
For a \verb|talk|, \verb|\separator| draws a horizontal line and \verb|\tosay{message}| prints \verb|message| inside a box as a reminder.
In all styles, \verb|\mymarginpar{message}| behaves like a regular \verb|marginpar|, but with a custom style.

\section{Licence}

\begin{quote}
``Please note that my class is not licensed in any way.
If you want to use it, you're free to do anything you want with them.
Anyway, I will be happy if you just drop me a line telling me you're using it.''
\\
---Stefano Maggiolo
\end{quote}

For my adaptation of Stefano Maggiolo's class I will be using a licence, but it will be an ``open'' licence in keeping with his original intentions.

\begin{quote}
Copyright \copyright\ 2016 Jahrme Risner.\\
Permission is granted to copy, distribute and/or modify this document
under the terms of the GNU Free Documentation License, Version 1.3
or any later version published by the Free Software Foundation;
with no Invariant Sections, no Front-Cover Texts, and no Back-Cover Texts.
A copy of the license is included in the section entitled "GNU
Free Documentation License".
\end{quote}

\end{document}
